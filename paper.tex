%%%%%%%%%%%%%%%%%%%%%%%%%%%%%%%%%%%%%%%%%%%%%%%%%%%%%%%%%%%%%%%%%%%%%%%%%%%%%
%%% LaTeX-Rahmen fuer das Erstellen von Diplomarbeiten
%%%%%%%%%%%%%%%%%%%%%%%%%%%%%%%%%%%%%%%%%%%%%%%%%%%%%%%%%%%%%%%%%%%%%%%%%%%%%

%%%%%%%%%%%%%%%%%%%%%%%%%%%%%%%%%%%%%%%%%%%%%%%%%%%%%%%%%%%%%%%%%%%%%%%%%%%%%
%%% allgemeine Einstellungen
%%%%%%%%%%%%%%%%%%%%%%%%%%%%%%%%%%%%%%%%%%%%%%%%%%%%%%%%%%%%%%%%%%%%%%%%%%%%%

\documentclass[twoside,12pt,a4paper]{report}
%\usepackage{reportpage}
\usepackage[utf8]{inputenc}
\usepackage{epsf,german}
\usepackage{graphics, graphicx}
\usepackage{latexsym}
\usepackage{amsfonts}
\usepackage{amsmath}
\usepackage{url}
\usepackage[margin=10pt,font=small,labelfont=bf]{caption}
\usepackage{xcolor}
\definecolor{darkred}{rgb}{0.5,0,0}
\definecolor{red}{rgb}{1,0,0}
\definecolor{darkgreen}{rgb}{0,0.5,0}
\definecolor{darkblue}{rgb}{0,0,0.5}
\usepackage{hyperref}
\hypersetup{colorlinks,
linkcolor=darkblue,
citecolor=darkgreen,
urlcolor=blue
}



\textwidth 14cm
\textheight 22cm
\topmargin 0.0cm
\evensidemargin 1cm
\oddsidemargin 1cm
%\footskip 2cm
\parskip0.5explus0.1exminus0.1ex

% Kann von Student auch nach persönlichem Geschmack verändert werden.
\pagestyle{headings}

\sloppy

\begin{document}

%%%%%%%%%%%%%%%%%%%%%%%%%%%%%%%%%%%%%%%%%%%%%%%%%%%%%%%%%%%%%%%%%%%%%%%%%%%%
%%% hier steht die neue Titelseite 
%%%%%%%%%%%%%%%%%%%%%%%%%%%%%%%%%%%%%%%%%%%%%%%%%%%%%%%%%%%%%%%%%%%%%%%%%%%%
 
\begin{titlepage}
	\begin{center}
  		{\LARGE Eberhard Karls Universität Tübingen}\\
		{\large Mathematisch-Naturwissenschaftliche Fakultät \\
		Wilhelm-Schickard-Institut für Informatik\\[4cm]}
  		{\huge Studienarbeit Informatik\\[2cm]}
  		{\Large\bf  Orientierungsberechnung mittels Multisensordatenfusion auf iOS-Endgeräten\\[1.5cm]}
 		{\large Sebastian Engel}\\[0.5cm]
		26.02.2012 \\[4cm]
		{\small\bf Betreuer}\\[0.5cm]
  		\parbox{7cm}{
  			\begin{center}
				{\large Jürgen Sommer}\\
	  			{\footnotesize Wilhelm-Schickard-Institut für Informatik\\
				Universität Tübingen}
			\end{center}
 		}
 	\end{center}
\end{titlepage}

%%%%%%%%%%%%%%%%%%%%%%%%%%%%%%%%%%%%%%%%%%%%%%%%%%%%%%%%%%%%%%%%%%%%%%%%%%%%
%%% Titelrückseite: Bibliographische Angaben
%%%%%%%%%%%%%%%%%%%%%%%%%%%%%%%%%%%%%%%%%%%%%%%%%%%%%%%%%%%%%%%%%%%%%%%%%%%%

\thispagestyle{empty}
\vspace*{\fill}
\begin{minipage}{11.2cm}
\textbf{Engel, Sebastian:}\\
\emph{Orientierung von Objekten bei inertialer Navigation}\\ Studienarbeit Informatik\\
Eberhard Karls Universität Tübingen\\
Bearbeitungszeitraum: 10/2011 - 01/2012
\end{minipage}
\newpage

%%%%%%%%%%%%%%%%%%%%%%%%%%%%%%%%%%%%%%%%%%%%%%%%%%%%%%%%%%%%%%%%%%%%%%%%%%%%

\pagenumbering{roman}
\setcounter{page}{1}

%%%%%%%%%%%%%%%%%%%%%%%%%%%%%%%%%%%%%%%%%%%%%%%%%%%%%%%%%%%%%%%%%%%%%%%%%%%%
%%% Seite I: Zusammenfassug, Danksagung
%%%%%%%%%%%%%%%%%%%%%%%%%%%%%%%%%%%%%%%%%%%%%%%%%%%%%%%%%%%%%%%%%%%%%%%%%%%%


\section*{Zusammenfassung}

Im Rahmen eines Navigations-Programms für Bibliotheken auf mobilen Geräten der neusten Generation beschäftigt sich diese Studienarbeit mit der Orientierung (Winkellage) des Geräts. Zur genauen Navigation in Räumen genügt die Bestimmung der Position nicht. Es ist zusätzlich relavant wie das Gerät orientiert ist. Durch diese Information ist es möglich den Benutzer direkt zu einem bestimmten Ort, im Falle einer Bibliothek ein Buch, zu führen. Im Wesentlichen befasst sich die Studienarbeit mit der Beschaffung und Berechnung der Orientierungsdaten.

\newpage
\section*{Danksagung}

Vielen Dank an Jürgen Sommer, Alex Decker, Stephan Doerr, Martin Lahl, Philipp Wolter, Markus ... und Sabrina Pfeffer.

\cleardoublepage

%%%%%%%%%%%%%%%%%%%%%%%%%%%%%%%%%%%%%%%%%%%%%%%%%%%%%%%%%%%%%%%%%%%%%%%%%%%%%
%%% Inhaltsverzeichnis
%%%%%%%%%%%%%%%%%%%%%%%%%%%%%%%%%%%%%%%%%%%%%%%%%%%%%%%%%%%%%%%%%%%%%%%%%%%%%

\renewcommand{\baselinestretch}{1.3}
\small\normalsize

\tableofcontents

\renewcommand{\baselinestretch}{1}
\small\normalsize

\cleardoublepage

%%%%%%%%%%%%%%%%%%%%%%%%%%%%%%%%%%%%%%%%%%%%%%%%%%%%%%%%%%%%%%%%%%%%%%%%%%%%%
%%% Abbildungsverzeichnis
%%%%%%%%%%%%%%%%%%%%%%%%%%%%%%%%%%%%%%%%%%%%%%%%%%%%%%%%%%%%%%%%%%%%%%%%%%%%%

\renewcommand{\baselinestretch}{1.3}
\small\normalsize

\addcontentsline{toc}{chapter}{Abbildungsverzeichnis}
\listoffigures

\renewcommand{\baselinestretch}{1}
\small\normalsize

\cleardoublepage

%%%%%%%%%%%%%%%%%%%%%%%%%%%%%%%%%%%%%%%%%%%%%%%%%%%%%%%%%%%%%%%%%%%%%%%%%%%%%
%%% Tabellenverzeichnis
%%%%%%%%%%%%%%%%%%%%%%%%%%%%%%%%%%%%%%%%%%%%%%%%%%%%%%%%%%%%%%%%%%%%%%%%%%%%%

\renewcommand{\baselinestretch}{1.3}
\small\normalsize

\addcontentsline{toc}{chapter}{Tabellenverzeichnis}
\listoftables

\renewcommand{\baselinestretch}{1}
\small\normalsize

\cleardoublepage

%%%%%%%%%%%%%%%%%%%%%%%%%%%%%%%%%%%%%%%%%%%%%%%%%%%%%%%%%%%%%%%%%%%%%%%%%%%%%
%%% Abkürzungsverzeichnis
%%%%%%%%%%%%%%%%%%%%%%%%%%%%%%%%%%%%%%%%%%%%%%%%%%%%%%%%%%%%%%%%%%%%%%%%%%%%%

\addcontentsline{toc}{chapter}{Abkürzungsverzeichnis}
\chapter*{Abkürzungsverzeichnis\markboth{ABKÜRZUNGSVERZEICHNIS}{ABKÜRZUNGSVERZEICHNIS}}

\begin{tabbing}
\textbf{FACTOTUM}\hspace{1cm}\=Schrott\kill
\textbf{IMU}\>Inertial Measuring Unit \\
\textbf{API} \> Application Programming Interface\\
\textbf{GPS} \> Global Positioning System\\
\textbf{MEMS} \> Microelectromechanical Systems\\
\textbf{} \> \\
\textbf{} \> \\
\end{tabbing}

\cleardoublepage

%%%%%%%%%%%%%%%%%%%%%%%%%%%%%%%%%%%%%%%%%%%%%%%%%%%%%%%%%%%%%%%%%%%%%%%%%%%%%
%%% Der Haupttext, ab hier mit arabischer Numerierung
%%% Mit \input{dateiname} werden die Datei `dateiname' eingebunden
%%%%%%%%%%%%%%%%%%%%%%%%%%%%%%%%%%%%%%%%%%%%%%%%%%%%%%%%%%%%%%%%%%%%%%%%%%%%%

\pagenumbering{arabic}
\setcounter{page}{1}

%% Introduction
%%%%%%%%%%%%%%%%%%%%%%%%%%%%%%%%%%%%%%%%%%%%%%%%%%%%%%%%%%%%%%%%%%%%
% Einleitung
%%%%%%%%%%%%%%%%%%%%%%%%%%%%%%%%%%%%%%%%%%%%%%%%%%%%%%%%%%%%%%%%%%%%

\chapter{Einleitung}\label{Einleitung}

\medskip
Die Arbeit gliedert sich wie folgt: Zu Beginn wird in Kapitel~\ref{Stand der Technik} die Entwicklung der benötigten Hard- und Software betrachtet. In Kapitel~\ref{Konzept} wird ein Konzept zur Berechnung der Orientierung eines mobilen Geräts entwickelt. Bevor sich der genauen Implementierung, am Beispiel einer Navigations-App für Bibliotheken, in Kapitel~\ref{Umsetzung} gewidmet wird, werden in Kapitel~\ref{Plattform und Werkzeuge} die für diese Herangehensweise geeignete Plattform und Werzkeuge gewählt, die zur Umsetzung benötigt werden. Die Umsetzung erfolgt auf Basis der eingebauten Sensoren. Es soll keine Zusatzhardware zum Einsatz kommen. Es folgt eine Auswertung der Ergebnisse in Kapitel~\ref{Ergebnis} mit einer Diskussion. Am Ende beschließt ein Ausblick in Kapitel~\ref{Ausblick} diese Arbeit. Die Studienarbeit entwickelt eine Diplomarbeit von Achim Fritz aus dem Jahr 2011 weiter.

\section{Motivation}\label{Motivation}
Bisher findet Navigation hauptsächlich im Freien statt – beispielsweise schon seit Langem bei Navigationssystemen für Autos. Dabei wird ausschließlich GPS verwendet. Für Navigation innerhalb von Gebäuden ist GPS aber nicht brauchbar. Es ist zu ungenau und wird durch die Wände des Gebäudes noch zusätzlich ungenauer. Bei der genauen Positionsbestimmung wird es zunehmend wichtiger, auch die Orientierung zu bestimmen. Denn innerhalb von Gebäuden und bei Positionsunterschieden von wenigen Metern ist die Information, in welche Richtung man schaut, ebenfalls interessant und liefert zusätzliche Informationen. Gerade bei einer Navigations-App für Bibliotheken ist es wichtig zu wissen, welches Regal im Moment angeschaut wird. Außerdem darf die Position eine Ungenauigkeit von einigen Zentimetern nicht überschreiten. Ziel dieser Arbeit ist es die Lage des Geräts im Raum zu berechnen. Die aus den Werten der Sensoren des Geräts berechneten Werte sollen für eine bereits bestehende praxistaugliche Navigations-App für Bibliotheken die Orientierungs-Werte beisteuern.
\cleardoublepage

%% 
%%%%%%%%%%%%%%%%%%%%%%%%%%%%%%%%%%%%%%%%%%%%%%%%%%%%%%%%%%%%%%%%%%%%
% Grundlagen
%%%%%%%%%%%%%%%%%%%%%%%%%%%%%%%%%%%%%%%%%%%%%%%%%%%%%%%%%%%%%%%%%%%%

\chapter{Stand der Technik}
  \label{Stand der Technik}

\medskip
Bisher findet Navigation hauptsächlich im Freien statt. Zum Beispiel schon seit langem bei Navigationssystemen für Autos. Dabei wird ausschließlich GPS verwendet. Für Navigation innerhalb von Gebäuden ist GPS nicht brauchbar. Es ist zu ungenau und wird durch die Wände des Gebäudes noch zusätzlich ungenauer. Bei der genauen Positionsbestimmung wird es zunehmend wichtiger auch die Orientierung zu bestimmen. Denn innerhalb von Gebäuden und bei Positionsunterschieden von wenigen Metern ist die Information in welche Richtung man schaut ebenfalls interessant und liefert zusätzliche Information. Gerade bei einer Navigations-App für Bibliotheken ist es wichtig zu wissen welches Regal gerade angeschaut wird. Außerdem darf die Position eine Ungenauigkeit von einigen Zentimetern nicht überschreiten, denn sonst ist die Führung zum Buch so ungenau, dass die App mehr Umstand als Nutzen bringt.

\section{Beschreibung der Orientierung von Objekten im dreidimensionalen Raum}
Zur Beschreibung der Orientierung von Objekten im dreidimensionalen Raum in kartesischen Koordinatensystemen gibt es mehrere Möglichkeiten. Die drei am häufigsten verwendeten und für uns relevanten werden im Folgenden vorgestellt.

\subsection{Euler-Winkel}
Bei Euler-Winkeln handelt es sich um drei Winkel die jeweils die Rotation um eine bestimmte Achse des Koordinatensystems angeben. So kann eine Transformation zwischen zwei Koordinatensystemen, dem Labor- und dem Körperfesten-System definiert werden.

Es existieren mehrere Definitionen von Euler-Winkeln, was die Reihenfolge der Drehungen um die Achsen anbelangt. Für unsere Zwecke beschäftigen wir uns mit Yaw-Pitch-Roll - zu deutsch: Roll-Nick-Gier-Winkel. Dies entspricht auch der Luftfahrtnorm (DIN 9300).

\begin{itemize}
	\item Roll (Roll-Winkel) beschreibt die Querneigung, also die Drehung um die X-Achse.
	\item Pitch (Nick-Winkel) besschreibt die Längsneigung, also die Drehung um die Y-Achse.
	\item Yaw (Gier-Winkel) beschreibt Orientierung, also die Drehung um die Z-Achse.
\end{itemize}

Bei mobilen Geräten wie dem Apple iPhone gibt es anders als bei Fahrzeugen keine fest denifierte Ausrichtung. Beim iPhone und iPad sind die Winkel darum so verteilt wie auf Bild \ref{fig:apple-axes} zu sehen.

\begin{figure}[htb]
\centering
\includegraphics[scale=0.8]{figures/apple-axes}
\caption{Roll-Pitch-Yaw \cite{apple:001}}
\label{fig:apple-axes}
\end{figure}

Euler-Winkel haben den Vorteil, dass sie jeder intuitiv verstehen kann. Jeder lernt Euler-Winkel in der Schule kennen. Somit kann man mit ihnen auch einfach rechnen.

Eine Drehung mit Euler-Winkeln setzt sich immer aus einer Kombinationen von Rotation der drei Achsen zusammen. Das heißt eine Drehung findet nie direkt statt, sondern über mehrere nacheinander ausgeführte Rotationen der einzelnen Achsen. Das kann bei manchen Anwendungen ein Problem sein. Zum Beispiel wenn die Orientierung schneller abgefragt wird als die Teilschritte einer Drehung berechnet werden, kann es vorkommen, dass in einzelnen Key-Frames der Anwendung falsche Orientierungsdaten einfließen.

\subsubsection{Gimbal Lock}
Der große Nachteil von Euler-Winkeln ist der Gimbal Lock (engl. f. Blockade der Kardanischen Aufhängung). So nennt man es wenn zwei Achsen die selbe Drehung bestimmen. Dadurch fehlt ein Freiheitsgrad. Man kann eine bestimmte Drehung erst dann wieder durchführen wenn man eine der beiden zusammengefallenen Achsen zurück dreht. In Abbildung \ref{fig:gimbal-lock} ist leicht zu erkennen, dass die innere (blau) und äußere (grün) Achse die selbe Drehung bestimmen. Es ist daher momentan nicht möglich das Flugzeug nach vorne oder hinten zu Kippen. Erst müsste das Flugzeug entlang der mittleren Achse um $90^o$ gedreht werden. \cite{ wiki:002} \cite{paper:001}

\begin{figure}[htb]
\centering
\includegraphics[scale=0.4]{figures/gimbal-lock}
\caption{Gimbal Lock \cite{ wiki:004} }
\label{fig:gimbal-lock}
\end{figure}

\subsection{Rotationsmatrizen}
Eine Rotationsmatrix ist eine orthogonale Matrix, die ebenfalls die Drehung im Raum beschreibt. Sie ist als eine Hintereinanderausführung einer oder mehrerer Rotationen um beliebige Drehachsen im dreidimensionalen Raum definiert. Darum ist klar, dass Rotationsmatrizen quasi nur eine Zusammenfassung von einzelnen Rotationen mit Euler-Winkeln in ein einziges Kontrukt sind. Gimbal Lock kann auch mit Rotationsmatrizen auftreten. \cite{ wiki:003}

\subsection{Quaternionen}
Quaternionen sind ein mathematisches Konstrukt um Orientierung von Objekten im dreidimensionalen raum zu beschreiben. Sie setzen sich aus einem skalaren und einem vektoriellen Teil zusammen. Der vektorielle Teil ist allein dazu da um die Achse der durchzuführenden Drehung zu beschreiben. Der Skalaranteil gibt den Winkel der Drehung an. Es wird also für jede Rotation eine eigene Achse kontruiert entlang der gedreht wird. Dadurch gibt es keine Zwischenschritte, sondern nur eine einzige Rotation. So kann auch das Problem des Gimbal Locks garnicht erst entstehen.

\section{Positionsbestimmung}
Die Positionsbestimmung erfolgt in unserem Fall über Bluetooth. Dazu werden in dem Raum in dem man Navigieren möchte Bluetooth-Sender (Beacons) ausgelegt. Diese sollten möglichst gleichmäßig verteilt sein, damit eine gleichmäßige Bluetooth-Abdeckung gewährleistet ist. Die App weiß wo sich die einzelnen Beacons im Raum befinden und empfängt die RSSI-Werte der ausgelegten Beacons und kann so die Position des Geräts bestimmen. 

Dabei können mehrere Probleme Auftreten. Das größte ist, die Senderate der Beacons. Herkömmliche Bluetooth-Sticks sind dafür gemacht eine ständige Verbindung zu einem Gerät aufrechtzuerhalten um Daten zu übertragen. Bei unserem Anwendungsfall wollen wir keine Daten übertragen, sondern nur möglichst oft den RSSI-Wert der einzelnen Beacons erfahren. Normale Bluetooth-Sticks schaffen meistens nur eine Rate von ca. drei Sekunden. Das ist zu wenig um mit wenigen Sticks eine zuverlässige Navigation zu realisieren. Mann muss entweder in relativ teure (über 100) Bluetooth-Sender investieren die schnell sind, oder viele von den langsameren auslegen.

Ein weiteres Problem ist, dass Bluetooth leicht gestört werden kann. Personen, Wände, Bücherregale sind ein Problem bei Bluetooth. Darum kann man nicht sicher sein, dass die RSSI-Werte die beim Gerät ankommen die entsprechende Entfernung repräsentieren. 

Um diesen beiden Hauptproblemen entgegenzuwirken wird ein Partikelfilter eingesetzt. Mit dem partikelfilter wird eine Wolke gewichteter Partikel erzeugt die den aktuellen Aufenthaltsort schätzen. Anhand der aktuellsten Position die aus den Bluetooth-RSSI-Werten berechnet wurde werden die einzelnen Partikel gewichtet. So kann die Ungenauigkeit der Bluetooth-Werte etwas korrigiert werden. \cite{wiki:001}
\cleardoublepage

%%
%%%%%%%%%%%%%%%%%%%%%%%%%%%%%%%%%%%%%%%%%%%%%%%%%%%%%%%%%%%%%%%%%%%%
% Eigener Ansatz
%%%%%%%%%%%%%%%%%%%%%%%%%%%%%%%%%%%%%%%%%%%%%%%%%%%%%%%%%%%%%%%%%%%%

\chapter{Eigener Ansatz}
  \label{Ansatz}
  
Die Orientierungs-Angabe, die man aus den Gyroskop-Daten gewinnt ist relativ zur Position des Geräts bei Beginn der Messung. Ohne zusätzliche absolute Angaben kann man nicht die absolute Orientierung im Raum bestimmen. Darum ist es von Nutzen zusätzlich den Kompass (und das Accelerometer) zu nutzen.
  
\section{Kompassstabilisierung mit Gyroskop}
Eine funktionierende Orientierung lässt sich bereits mit Kompass und dem Pitch-Wert den das Gyroskop liefert realisieren. 
















Eine Orientierungs-Angabe als Euler-Winkel würde beispielsweise wie folgt aussehen: 
$$
\begin{pmatrix}
    0.016134\\ 
    -0.000284\\ 
    1.618407
  \end{pmatrix} 
$$
Die Werte sind in Radiant angeben. Negative Werte können zustande kommen, da die Skala von $-\pi$ bis $+\pi$ geht.





\cleardoublepage

%%
%%%%%%%%%%%%%%%%%%%%%%%%%%%%%%%%%%%%%%%%%%%%%%%%%%%%%%%%%%%%%%%%%%%%
% Auswahl geeigneter Hardware
%%%%%%%%%%%%%%%%%%%%%%%%%%%%%%%%%%%%%%%%%%%%%%%%%%%%%%%%%%%%%%%%%%%%

\chapter{Plattform und Werkzeuge}
  \label{Plattform und Werkzeuge}
  
  \section{Plattform}
  
  \medskip
Bei der Suche nach einem passenden Gerät kamen mehere Kriterien zum Tragen. Es sollte wegen der Visualisierung größer als ein Smartphone sein, aber trotzdem portabler als ein herkömmliches Notebook. Es stand also fest, dass ein Tablet am besten geeignet ist für diese Art Anwendung. Desweiteren muss das Gerät mit den oben beschriebenen Sensoren, Accelerometer, Gyroskop und Kompass ausgestattet sein.

	\subsection{Überblick am Markt befindlicher Geräte}
Wirklich am Markt vertreten waren zum Zeitpunkt der Hardware-Entscheidung (Anfang 2011) nur das Apple iPad 1 und das Motorola Xoom. Das iPad war mit Accelerometer und Kompass ausgestattet, jedoch nicht mit einem Gyroskop. Das Motorola Xoom hatte alle drei IMUs verbaut. Android Version 3.0 Honeycomb erschien im Februar 2011 und war die erste Android-Version, die für Tablets ausgelegt war. \cite{ wiki:005} Allerdings war diese Version des Betriebssystems anfangs Berichten zufolge instabil.
	
	\subsection{Wahl iPad 2}
	Da das iPad 2 in den Startlöchern stand entschieden wir, unsere Entscheidung noch aufzuschieben. Am 11. März 2011 wurde es vorgestellt und die Ausstattung entsprach unseren Anforderungen, da zusätzlich ein Gyroskop verbaut wurde. Zum Erscheinungszeitpunkt war Apple auch der Hersteller mit der meisten Erfahrung. Das iPad der ersten Generation war bereits ein Jahr auf dem Markt und hatte mit iOS ein ausgereiftes Betriebssystem. iOS wird auf dem Apple iPhone schon seit 2007 verwendet. Zudem war das iPad damals der unangefochtene Marktführer. Damit war auch gewährleistet, dass für eine eventuell entstehende App für Endanwender genügend potentielle Abnehmer bereit stünden.
	

\section{Werkzeuge}

\subsection{Frameworks}
Zur Erstellung von iOS-Programmen steht Entwicklern Cocoa Touch von Apple zur Verfügung. Cocoa Touch ist eine Sammlung von Frameworks, die Entwickler bei der Programmierung unterstützen sollen. Grob lassen sich drei Arten von Frameworks, die in Cocoa Touch enthalten sind, nennen:

\begin{itemize}
\item Funktionen der Hardware
\item Design-Elemente und Animationen
\item Verarbeitung von Daten
\end{itemize}

Cocoa Touch ist in Objective-C implementiert. Somit wird in diesem Projekt fast ausschließlich Objective-C verwendet. \cite{apple:002}

Bei iOS sind für die Berechnung der Orientierung vor allem zwei Frameworks wichtig, die beide zum ersten Punkt der obigen Liste gehören: Core Location um den Kompass und Core Motion um das Gyroskop und das Accelerometer auszulesen.\\

\subsubsection{Core Motion}
Core Motion liefert Daten die, mit Bewegung zu tun haben. Das sind einerseits die rohen Daten aller drei Sensoren, die in Kapitel \ref{Konzept} beschrieben wurden, andererseits stellt Core Motion aber auch bereits bereinigte Bewegungs-Daten zur Verfügung. Zum Beispiel lassen sich Beschleunigung und Gravitation getrennt auslesen. Core Motion nimmt auch bereits die Stabilisierung des Elevation-Werts des Gyroskop vor. Besonders interessant ist die Klasse \texttt{CMAttitude} denn sie beinhaltet die Orientierung des Geräts zum Zeitpunkt der Abfrage. \texttt{CMAttitude} stellt diese in allen drei in Kapitel \ref{Stand der Technik} beschriebenen Formen zur Verfügung.

\subsubsection{Core Location}
Core Location enthält alle Informationen, die zur Bestimmung der aktuellen Position und der Ausrichtung des Geräts nötig sind. Core Location ist auch in der Lage, aus einem Geocode die zugehörige Stadt zu ermitteln und anders herum. Das \texttt{CMHeading}-Objekt stellt den aktuellen Kompass-Wert bereit.

\subsection{Unity}
Unity ist eine Spiele-Engine zur Entwicklung von Spielen für verschiedene Plattformen. Darunter sind PC-Betriebssyteme, mobile Betriebssyteme und Browser, aber auch herkömmliche Spiele-Konsolen. Unity ist neben der Unreal Engine eine der beliebtesten Spiele-Engines. In Unity kann man komplette 3D-Welten erstellen und es erlaubt den Export des Projekts in ein für die Zielplattform passendes Format. Im Falle von iOS wird ein Xcode Projekt erstellt. \cite{unity}

\subsection{Xcode}
Xcode ist die Entwicklungsumgebung für iOS-Anwendungen – hier wird der Quellcode geschrieben und das Interface angeordnet. Außerdem kann die App, an der momentan gearbeitet wird, mit Xcode direkt auf ein angeschlossenes iOS-Gerät kompiliert werden. Am Ende wird mit Xcode auch die Datei zur Einreichung in den AppStore erstellt. \cite{apple:003}
\cleardoublepage

%%
%%%%%%%%%%%%%%%%%%%%%%%%%%%%%%%%%%%%%%%%%%%%%%%%%%%%%%%%%%%%%%%%%%%%
% Umsetzung
%%%%%%%%%%%%%%%%%%%%%%%%%%%%%%%%%%%%%%%%%%%%%%%%%%%%%%%%%%%%%%%%%%%%

\chapter{Umsetzung}
  \label{Umsetzung}
  
  \section{Unity-Integration}
  Als Ausgangspunkt liegt eine voll funktionsfähige Navigations-App für Bibliotheken vor. Sie wurde komplett in Unity umgesetzt. Die Einbindung der Orientierungsberechnung kann nicht direkt in Unity vorgenommen werden, da in Unity nur ein eingeschränkter Zugriff auf die Cocoa-Frameworks möglich ist. Die App wird als Xcode Projekt exportiert. Dann muss in Xcode ein Plug-in Programmiert werden, das als Schnittstelle zwischen selbst geschriebenem Objective-C-Code in Xcode und Unity dient. Dazu muss erst noch in Unity das Plug-in als Eingabe-Skript erstellt werden. Dieses Eingabe-Skript ist in C\# geschrieben.
~\\
\begin{lstlisting}[float=htb, caption=Plug-in in Unity]
[DllImport ("__Internal")]
private static extern CMQuaternion getDeviceMotion();
\end{lstlisting}
  
Hier wird also die externe Methode \texttt{getDeviceMotion()} aufgerufen. Das Plug-in erwartet den Datentyp \texttt{CMQuaternion}. Wir übergeben später einen Quaternion, also immer die aktuelle absolute Orientierung. \texttt{CMQuaternion} ist eigentlich ein Datentyp auf dem Core Motion Framework und somit Unity nicht bekannt. Darum musste erst noch ein \texttt{struct} mit der Bezeichung \texttt{CMQuaternion} definieren.
~\\
\begin{lstlisting}[float=htb, caption=Struct \texttt{CMQuaternion} in C\#]
public struct CMQuaternion {
	public double x, y, z, w;
}
\end{lstlisting}

Alles weitere erfolgt jetzt direkt in Xcode. In Xcode wird eine \texttt{*.mm}-Datei angelegt zusammen mit der zugehörigen \texttt{*.h}-Datei. In der \texttt{*.mm}-Datei erfolgt die ganze Implementierung. Damit in der bestehenden App überhaupt Daten die hier berechnet werde ankommen muss als erstes die \texttt{getDeviceMotion()}-Methode implementiert werden:
~\\
\begin{lstlisting}[float=htb, caption=Methode \texttt{getDeviceMotion}]
static GyroscopeData* delegateObject = nil;

extern "C" {

	CMQuaternion getDeviceMotion () {

		if (delegateObject == nil) {
			delegateObject = [[GyroscopeData alloc] init];
		}
		
		return [delegateObject getOrientation];(*@\label{line001}@*)
	}    
}
\end{lstlisting}

Allerdings muss in dieser \texttt{extern  C }-Umgebung in C geschrieben werden. In C sind aber die API-Aufrufe nicht oder nur sehr umständlich möglich. Darum wird in Zeile \ref{line001} eine weitere Methode \texttt{getOrientation} aufgerufen. 
~\\
\begin{lstlisting}[float=htb, caption=Methode \texttt{getOrientation}]
- (CMQuaternion)getOrientation {
	...
	...
}
\end{lstlisting}

Diese Methode enthält den eigentlichen Objective-C-Quellcode. In ihr können alle API-Aufrufe problemlos ausgeführt werden.

\cite{unity}

  
\section{Vorbereitung der nötigen Daten}
Um die Multidatenfusion durchführen zu können müssen erst alle nötigen Daten auslegesen werden. Dazu müssen \texttt{CMMotionManager}- und \texttt{CLLocationManager}-Objekte initialisert werden. 
~\\
\begin{lstlisting}[float=htb, caption=\texttt{locationManager} und \texttt{motionManager} initialisieren \cite{apple:003}]
// Set up locationManager
if (locationManager == nil) {
	locationManager=[[CLLocationManager alloc] init];
	locationManager.desiredAccuracy = kCLLocationAccuracyBest;(*@\label{line002}@*)
}
    
// Set up motionManager    
if (motionManager == nil) {
	motionManager = [[CMMotionManager alloc] init];
	motionManager.deviceMotionUpdateInterval = 1.0/60.0;(*@\label{line003}@*)
}
\end{lstlisting}

In Zeile \ref{line002} wird die Genauigkeit des \texttt{CLLocationManager}-Objekts und in Zeile \ref{line003} die Update-Frequenz des \texttt{CMMotionManager}-Objekts eingestellt. Der Kompass-Wert den wir aus dem \texttt{CLLocationManager}-Objekt auslesen muss sehr genau sein.


Das Auslesen des Kompass-Werts findet eventgesteuert statt. Ein neuer Wert wird nur dann ausgelesen wenn er sich vom alten Wert unterscheidet. Dazu setzen wir die minimale Winkeländerung auf $1^o$ fest.
~\\
\begin{lstlisting}[float=htb, caption=\texttt{locationManager} starten \cite{apple:003}]
// Start listening to events from locationManager
if ([CLLocationManager headingAvailable]) {
	locationManager.headingFilter = 1;
	[locationManager startUpdatingHeading];(*@\label{line004}@*)
}
\end{lstlisting}

Mit dem Aufrufen der Methode \texttt{startUpdatingHeading} in Zeile \ref{line004} wird hier auch gleichzeitig das eventgesteuerte Abfragen des Kompass-Werts gestartet.

Die Methode die auf die Kompass-Änderungen hört ließt den Kompass-Wert aus und stellt ihn in einer globalen Variable zur Verfügung.
~\\
\begin{lstlisting}[float=htb, caption=Azimut ermittelt durch Kompass]
- (void)locationManager:(CLLocationManager *)manager didUpdateHeading:(CLHeading *)newHeading {
	// Get new heading
	mHeading = newHeading.magneticHeading;    
    
	//location specific offset depending on the 3D model
	locationOffset = 90;
	mHeading += locationOffset;
    
	if (mHeading > 360) {
		mHeading -= 360;
	}
	else if (mHeading < 0) {
		mHeading += 360;
	}
}
\end{lstlisting}

Es kann vorkommen, dass in dem 3D-Modell des Raums nicht an der selben Stelle Norden ist wie in der Realität an diesem Ort. Darum muss man, wenn dieser Fall auftritt, einen Offset mit dem ausgelesenen Wert addieren. Das Resultat kann ein Wert sein der entweder größer als $360^o$ oder kleiner als $0^o$ ist. Der Wert muss dann normalisiert werden indem $360^o$ subtrahiert oder addiert werden. Jetzt haben wir die Azimut ermittelt durch den Kompass.

Die Azimut-Änderung ermittelt duch das Gyroskop und die Elevation liefert das \texttt{CMMotionManager}-Objekt.
~\\
\begin{lstlisting}[float=htb, caption=Bewegungsdaten auslesen \cite{apple:003}]
if(motionManager.isDeviceMotionAvailable) {
        
	// Listen to events from the motionManager
	motionHandler = ^ (CMDeviceMotion *motion, NSError *error) {
	
	CMAttitude *currentAttitude = motion.attitude;(*@\label{line005}@*)
	.
	.
	.
}
\end{lstlisting}

Im \texttt{CMDeviceMotion}-Objekt werden Messungen des Accelerometers und des Gyroskops zusammengefasst. Der \texttt{motionHandler} wird immer dann aufgerufen, wenn es Bewegungs-Daten des Geräts zu verarbeiten gibt. Hier ist das alle 1/60 Sekunden der Fall, weil wir das bei der Initialisierung des \texttt{CMMotionManager}-Objekts so festgelegt haben.

Das auslesen der eigentlichen Orientierungsdaten erfolgt in Zeile \ref{line005}. Wobei in diesem \texttt{CMAttitude}-Objekt alle drei Beschreibungs-Möglichkeiten, Euler-Winkel, Rotationsmatrix und Quaternion, zusammengefasst sind.
~\\
\begin{lstlisting}[float=htb, caption=Azimut-Änderung berechnen]
quaternion = currentAttitude.quaternion;

if (oldQuaternion.w != 0 || oldQuaternion.x != 0 || oldQuaternion.y != 0 || oldQuaternion.z != 0){
	diffQuaternion = [self multiplyQuaternions:[self inverseQuaternion:oldQuaternion] :quaternion];
	diffQuaternion = [self normalizeQuaternion:diffQuaternion];
}            
oldQuaternion = quaternion;
\end{lstlisting}

Die Orientierung wird als Queternion ausgelesen und die Differenz zur letzten Orientierung gespeichert. Dies wird erreicht indem der inverse Quaternion des alten Werts mit dem Quaternion des neuen Werts multipliziert wird. Danach muss das Ergebnis noch normalisiert werden. Dazu wurden die vier Methoden \texttt{quaternionMagnitude}, \texttt{inverseQuaternion}, \texttt{multiplyQuaternions} und \texttt{normalizeQuaternion} geschrieben.
~\\
\begin{lstlisting}[float=htb, caption=Methode \texttt{quaternionMagnitude}]
- (float) quaternionMagnitude:(CMQuaternion)inputQuaternion {
	float magnitude = sqrt(inputQuaternion.w*inputQuaternion.w + inputQuaternion.x*inputQuaternion.x + inputQuaternion.y*inputQuaternion.y + inputQuaternion.z*inputQuaternion.z);
	
	return magnitude;
}
\end{lstlisting}


\begin{lstlisting}[float=htb, caption=Methode \texttt{inverseQuaternion}]
- (CMQuaternion) inverseQuaternion:(CMQuaternion)inputQuaternion {
	float magnitude = [self quaternionMagnitude:inputQuaternion];
	
	quaternion.w = inputQuaternion.w/magnitude;
	quaternion.x = -inputQuaternion.x/magnitude;
	quaternion.y = -inputQuaternion.y/magnitude;
	quaternion.z = -inputQuaternion.z/magnitude;
	
	return quaternion;
}
\end{lstlisting}

\begin{lstlisting}[float=htb, caption=Methode \texttt{multiplyQuaternions}]
- (CMQuaternion) multiplyQuaternions:(CMQuaternion)quaternionA:(CMQuaternion)quaternionB {
	quaternion.w = quaternionA.w*quaternionB.w - quaternionA.x*quaternionB.x - quaternionA.y*quaternionB.y - quaternionA.z*quaternionB.z;
	quaternion.x = quaternionA.w*quaternionB.x + quaternionA.x*quaternionB.w - quaternionA.y*quaternionB.z + quaternionA.z*quaternionB.y;
	quaternion.y = quaternionA.w*quaternionB.y + quaternionA.x*quaternionB.z + quaternionA.y*quaternionB.w - quaternionA.z*quaternionB.x;
	quaternion.z = quaternionA.w*quaternionB.z - quaternionA.x*quaternionB.y + quaternionA.y*quaternionB.x + quaternionA.z*quaternionB.w;

	return quaternion;
}
\end{lstlisting}

\begin{lstlisting}[float=htb, caption=Methode \texttt{normalizeQuaternion}]
- (CMQuaternion) normalizeQuaternion:(CMQuaternion)inputQuaternion {
	float magnitude = [self quaternionMagnitude:inputQuaternion];
	
	quaternion.w = inputQuaternion.w / magnitude;
	quaternion.x = inputQuaternion.x / magnitude;
	quaternion.y = inputQuaternion.y / magnitude;
	quaternion.z = inputQuaternion.z / magnitude;
	
	return quaternion;
}
\end{lstlisting}

Interessant für uns sind Azimut und Elevation. Darum berechnen wir die beiden Werte in Grad aus dem Quaternion. Die Methoden \texttt{azimutFromQuaternion} und \texttt{elevationFromQuaternion} berechnen den entsprechenden Winkel in Radian.
~\\
\begin{lstlisting}[float=htb, caption=Azimut-Wert aus Quaternion berechnen]
- (float) azimutFromQuaternion:(CMQuaternion)quaternion {
	float azimut = atan2(2*(quaternion.w*quaternion.z+quaternion.x*quaternion.y), 1 - 2*(quaternion.y*quaternion.y+quaternion.z*quaternion.z));
	return azimut;
}

azimutDiff = RADIANS_TO_DEGREES([self azimutFromQuaternion:diffQuaternion]);
\end{lstlisting}

\begin{lstlisting}[float=htb, float=htb, caption=Elevation-Wert aus Quaternion berechnen, label=listing001]
- (float) elevationFromQuaternion:(CMQuaternion)quaternion {
	float elevation = atan2(2*(quaternion.w * quaternion.x + quaternion.y * quaternion.z), 1-2 * (quaternion.x * quaternion.x + quaternion.y * quaternion.y));
	return elevation;
} 

elevation = -[self elevationFromQuaternion:quaternion];
elevation += M_PI/2;
elevation = RADIANS_TO_DEGREES(elevation);
\end{lstlisting}

Bei der Elevation muss noch eine Korrektur von $90^o$ vorgenommen werden. Denn wenn man das iPad mit dem Display nach oben um $90^o$ neigt, so dass das Display zum Betrachter zeigt, befindet sich standardmäßig genau in dieser Position der Sprung von $0^o$ auf $360^o$. Um eventuellen problemen mit dieser Tatsache aus dem Weg zu gehen verschieben wir den Sprung um $90^o$ nach oben. Dann tritt er nur auf wenn der Betrachter das iPad direkt an die Decke hält.

Jetzt haben wir die Azimut-Änderung und die Elevation, basierend auf Gyroskop-Daten errechnet.

\cite{paper:001} \cite{wiki:007} \cite{mathworks} \cite{book001} \cite{book002}

\section{Multisensordatenfusion}
Hier müssen wir uns, danke Core Motion nur noch um den Azimut-Wert kümmern. Core Motion hat den Elevation-Wert aus dem Gyroskop bereits mit dem Accelerometer stabilisiert. Darum weißt der in Listing \ref{listing001} berechnete Elevation-Wert keinen bemerkbaren Drift mehr auf.

Nun können wir die Formel aus Kapitel \ref{formula001} umsetzen.
~\\
\begin{lstlisting}[float=htb, caption=Eigentliche Datenfusion, label=listing016]
updatedAzimut = updatedAzimut - azimutDiff;(*@\label{line006}@*)

float alpha = 19.0;
float phi = 1.0;

//fusionate gyro estimated heading with new magneticHeading
updatedAzimut = (alpha* updatedAzimut + phi*heading)/(alpha+phi);(*@\label{line007}@*)         
\end{lstlisting}

In Zeile \ref{line006} aus Listing \ref{listing016} wird die Azimut-Änderung auf den vorherigen Azimut-Wert angwendet. Zeile \ref{line007} ist die genaue Umsetzung der Formel ausl \ref{formula001}. \texttt{updatedAzimut} ist der vorherige zur Drehung verwendete Azimut-Wert korrigiert um die Drehung seit dem letzten Wert. \texttt{heading} ist der durch den Kompass bestimmte Azimut-Wert. Mit den beiden Steuerungsvariablen \texttt{alpha} und \texttt{phi} wird gesteuert welche Anteile die jeweiligen Komponenten der Fusion haben.
           

\cleardoublepage

%%
%%%%%%%%%%%%%%%%%%%%%%%%%%%%%%%%%%%%%%%%%%%%%%%%%%%%%%%%%%%%%%%%%%%%
% Ergebnis
%%%%%%%%%%%%%%%%%%%%%%%%%%%%%%%%%%%%%%%%%%%%%%%%%%%%%%%%%%%%%%%%%%%%

\chapter{Ergebnis}
  \label{Ergebnis}



\cleardoublepage

%%
\chapter{Ausblick}
  \label{Ausblick}

NFC \cite{wiki:006}, iPad 3, Android, Kalman-Filter, Bewegungsberechnung, Augmented Reality

Mit das Wichtigste nat"urlich!

Hier gilt es beides, die Info-Seite der Arbeit sowie die Bio-Seite zu diskutieren!!

Take your time for writing the discussion, it is the most important chapter of your thesis.
\clearpage
Mindestens 5 Seiten lang.
\clearpage
Ausblick kann auch ein extra Kapitel werden, wenn man das will.
\cleardoublepage


%%%%%%%%%%%%%%%%%%%%%%%%%%%%%%%%%%%%%%%%%%%%%%%%%%%%%%%%%%%%%%%%%%%%%%%%%%%%%
%%% Bibliographie
%%%%%%%%%%%%%%%%%%%%%%%%%%%%%%%%%%%%%%%%%%%%%%%%%%%%%%%%%%%%%%%%%%%%%%%%%%%%%

\addcontentsline{toc}{chapter}{Literaturverzeichnis}

\bibliographystyle{alpha}
\bibliography{mylit}
%% Obige Anweisung legt fest, dass BibTeX-Datei `mylit.bib' verwendet
%% wird. Hier koennen mehrere Dateinamen mit Kommata getrennt aufgelistet
%% werden.

\cleardoublepage

\end{document}

