%%%%%%%%%%%%%%%%%%%%%%%%%%%%%%%%%%%%%%%%%%%%%%%%%%%%%%%%%%%%%%%%%%%%
% Einleitung
%%%%%%%%%%%%%%%%%%%%%%%%%%%%%%%%%%%%%%%%%%%%%%%%%%%%%%%%%%%%%%%%%%%%

\chapter{Einleitung}\label{Einleitung}

\medskip
Die Arbeit gliedert sich wie folgt: Zu Beginn wird in Kapitel~\ref{Stand der Technik} die Entwicklung der benötigten Hard- und Software betrachtet. In Kapitel~\ref{Konzept} wird ein Konzept zur Berechnung der Orientierung eines mobilen Geräts entwickelt. Bevor sich der genauen Implementierung, am Beispiel einer Navigations-App für Bibliotheken, in Kapitel~\ref{Umsetzung} gewidmet wird, werden in Kapitel~\ref{Plattform und Werkzeuge} die für diese Herangehensweise geeignete Plattform und Werzkeuge gewählt, die zur Umsetzung benötigt werden. Die Umsetzung erfolgt auf Basis der eingebauten Sensoren. Es soll keine Zusatzhardware zum Einsatz kommen. Es folgt eine Auswertung der Ergebnisse in Kapitel~\ref{Ergebnis} mit einer Diskussion. Am Ende beschließt ein Ausblick in Kapitel~\ref{Ausblick} diese Arbeit. Die Studienarbeit entwickelt eine Diplomarbeit von Achim Fritz aus dem Jahr 2011 weiter.

\section{Motivation}\label{Motivation}
Bisher findet Navigation hauptsächlich im Freien statt – beispielsweise schon seit Langem bei Navigationssystemen für Autos. Dabei wird ausschließlich GPS verwendet. Für Navigation innerhalb von Gebäuden ist GPS aber nicht brauchbar. Es ist zu ungenau und wird durch die Wände des Gebäudes noch zusätzlich ungenauer. Bei der genauen Positionsbestimmung wird es zunehmend wichtiger, auch die Orientierung zu bestimmen. Denn innerhalb von Gebäuden und bei Positionsunterschieden von wenigen Metern ist die Information, in welche Richtung man schaut, ebenfalls interessant und liefert zusätzliche Informationen. Gerade bei einer Navigations-App für Bibliotheken ist es wichtig zu wissen, welches Regal im Moment angeschaut wird. Außerdem darf die Position eine Ungenauigkeit von einigen Zentimetern nicht überschreiten.