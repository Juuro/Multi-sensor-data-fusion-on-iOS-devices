%%%%%%%%%%%%%%%%%%%%%%%%%%%%%%%%%%%%%%%%%%%%%%%%%%%%%%%%%%%%%%%%%%%%
% Grundlagen
%%%%%%%%%%%%%%%%%%%%%%%%%%%%%%%%%%%%%%%%%%%%%%%%%%%%%%%%%%%%%%%%%%%%

\chapter{Stand der Technik}
  \label{Stand der Technik}

\medskip
Bisher findet Navigation hauptsächlich im Freien statt. Zum Beispiel schon seit langem bei Navigationssystemen für Autos. Dabei wird ausschließlich GPS verwendet. Für Navigation innerhalb von Gebäuden ist GPS nicht brauchbar. Es ist zu ungenau und wird durch die Wände des Gebäudes noch zusätzlich ungenauer. Bei der genauen Positionsbestimmung wird es zunehmend wichtiger auch die Orientierung zu bestimmen. Denn innerhalb von Gebäuden und bei Positionsunterschieden von wenigen Metern ist die Information in welche Richtung man schaut ebenfalls interessant und liefert zusätzliche Information. Gerade bei einer Navigations-App für Bibliotheken ist es wichtig zu wissen welches Regal gerade angeschaut wird. Außerdem darf die Position eine Ungenauigkeit von einigen Zentimetern nicht überschreiten, denn sonst ist die Führung zum Buch so ungenau, dass die App mehr Umstand als Nutzen bringt.

\section{Beschreibung der Orientierung von Objekten im dreidimensionalen Raum}
Zur Beschreibung der Orientierung von Objekten im dreidimensionalen Raum in kartesischen Koordinatensystemen gibt es mehrere Möglichkeiten. Die drei am häufigsten verwendeten und für uns relevanten werden im Folgenden vorgestellt.

\subsection{Euler-Winkel}
Bei Euler-Winkeln handelt es sich um drei Winkel die jeweils die Rotation um eine bestimmte Achse des Koordinatensystems angeben. So kann eine Transformation zwischen zwei Koordinatensystemen, dem Labor- und dem Körperfesten-System definiert werden.

Es existieren mehrere Definitionen von Euler-Winkeln, was die Reihenfolge der Drehungen um die Achsen anbelangt. Für unsere Zwecke beschäftigen wir uns mit Yaw-Pitch-Roll - zu deutsch: Roll-Nick-Gier-Winkel. Dies entspricht auch der Luftfahrtnorm (DIN 9300).

\begin{itemize}
	\item Roll (Roll-Winkel) beschreibt die Querneigung, also die Drehung um die X-Achse.
	\item Pitch (Nick-Winkel) besschreibt die Längsneigung, also die Drehung um die Y-Achse.
	\item Yaw (Gier-Winkel) beschreibt Orientierung, also die Drehung um die Z-Achse.
\end{itemize}

Bei mobilen Geräten wie dem Apple iPhone gibt es anders als bei Fahrzeugen keine fest denifierte Ausrichtung. Beim iPhone und iPad sind die Winkel darum so verteilt wie auf Bild \ref{fig:apple-axes} zu sehen.

\begin{figure}[htb]
\centering
\includegraphics[scale=0.8]{figures/apple-axes}
\caption{Roll-Pitch-Yaw \cite{apple:001}}
\label{fig:apple-axes}
\end{figure}

Euler-Winkel haben den Vorteil, dass sie jeder intuitiv verstehen kann. Jeder lernt Euler-Winkel in der Schule kennen. Somit kann man mit ihnen auch einfach rechnen.

Eine Drehung mit Euler-Winkeln setzt sich immer aus einer Kombinationen von Rotation der drei Achsen zusammen. Das heißt eine Drehung findet nie direkt statt, sondern über mehrere nacheinander ausgeführte Rotationen der einzelnen Achsen. Das kann bei manchen Anwendungen ein Problem sein. Zum Beispiel wenn die Orientierung schneller abgefragt wird als die Teilschritte einer Drehung berechnet werden, kann es vorkommen, dass in einzelnen Key-Frames der Anwendung falsche Orientierungsdaten einfließen.

\subsubsection{Gimbal Lock}
Der große Nachteil von Euler-Winkeln ist der Gimbal Lock (engl. f. Blockade der Kardanischen Aufhängung). So nennt man es wenn zwei Achsen die selbe Drehung bestimmen. Dadurch fehlt ein Freiheitsgrad. Man kann eine bestimmte Drehung erst dann wieder durchführen wenn man eine der beiden zusammengefallenen Achsen zurück dreht. In Abbildung \ref{fig:gimbal-lock} ist leicht zu erkennen, dass die innere (blau) und äußere (grün) Achse die selbe Drehung bestimmen. Es ist daher momentan nicht möglich das Flugzeug nach vorne oder hinten zu Kippen. Erst müsste das Flugzeug entlang der mittleren Achse um $90^o$ gedreht werden. \cite{ wiki:002} \cite{paper:001}

\begin{figure}[htb]
\centering
\includegraphics[scale=0.4]{figures/gimbal-lock}
\caption{Gimbal Lock \cite{ wiki:004} }
\label{fig:gimbal-lock}
\end{figure}

\subsection{Rotationsmatrizen}
Eine Rotationsmatrix ist eine orthogonale Matrix, die ebenfalls die Drehung im Raum beschreibt. Sie ist als eine Hintereinanderausführung einer oder mehrerer Rotationen um beliebige Drehachsen im dreidimensionalen Raum definiert. Darum ist klar, dass Rotationsmatrizen quasi nur eine Zusammenfassung von einzelnen Rotationen mit Euler-Winkeln in ein einziges Kontrukt sind. Gimbal Lock kann auch mit Rotationsmatrizen auftreten. \cite{ wiki:003}

\subsection{Quaternionen}
Quaternionen sind ein mathematisches Konstrukt um Orientierung von Objekten im dreidimensionalen raum zu beschreiben. Sie setzen sich aus einem skalaren und einem vektoriellen Teil zusammen. Der vektorielle Teil ist allein dazu da um die Achse der durchzuführenden Drehung zu beschreiben. Der Skalaranteil gibt den Winkel der Drehung an. Es wird also für jede Rotation eine eigene Achse kontruiert entlang der gedreht wird. Dadurch gibt es keine Zwischenschritte, sondern nur eine einzige Rotation. So kann auch das Problem des Gimbal Locks garnicht erst entstehen.

\section{Positionsbestimmung}
Die Positionsbestimmung erfolgt in unserem Fall über Bluetooth. Dazu werden in dem Raum in dem man Navigieren möchte Bluetooth-Sender (Beacons) ausgelegt. Diese sollten möglichst gleichmäßig verteilt sein, damit eine gleichmäßige Bluetooth-Abdeckung gewährleistet ist. Die App weiß wo sich die einzelnen Beacons im Raum befinden und empfängt die RSSI-Werte der ausgelegten Beacons und kann so die Position des Geräts bestimmen. 

Dabei können mehrere Probleme Auftreten. Das größte ist, die Senderate der Beacons. Herkömmliche Bluetooth-Sticks sind dafür gemacht eine ständige Verbindung zu einem Gerät aufrechtzuerhalten um Daten zu übertragen. Bei unserem Anwendungsfall wollen wir keine Daten übertragen, sondern nur möglichst oft den RSSI-Wert der einzelnen Beacons erfahren. Normale Bluetooth-Sticks schaffen meistens nur eine Rate von ca. drei Sekunden. Das ist zu wenig um mit wenigen Sticks eine zuverlässige Navigation zu realisieren. Mann muss entweder in relativ teure (über 100) Bluetooth-Sender investieren die schnell sind, oder viele von den langsameren auslegen.

Ein weiteres Problem ist, dass Bluetooth leicht gestört werden kann. Personen, Wände, Bücherregale sind ein Problem bei Bluetooth. Darum kann man nicht sicher sein, dass die RSSI-Werte die beim Gerät ankommen die entsprechende Entfernung repräsentieren. 

Um diesen beiden Hauptproblemen entgegenzuwirken wird ein Partikelfilter eingesetzt. Mit dem partikelfilter wird eine Wolke gewichteter Partikel erzeugt die den aktuellen Aufenthaltsort schätzen. Anhand der aktuellsten Position die aus den Bluetooth-RSSI-Werten berechnet wurde werden die einzelnen Partikel gewichtet. So kann die Ungenauigkeit der Bluetooth-Werte etwas korrigiert werden. \cite{wiki:001}